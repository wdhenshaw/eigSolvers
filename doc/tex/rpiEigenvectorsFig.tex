% -------------------------------------------------------------------------------------------------
{
\newcommand{\width}{4cm}
\begin{figure}[htb]
\begin{center}
\begin{tikzpicture}
   \useasboundingbox (0,.5) rectangle (16,16);  % set the bounding box (so we have less surrounding white space)
  \begin{scope}[yshift=3*\width]   
    \figByWidthb{       0}{0}{fig/rpiEigenVector0}{\width}[0.08][0.11][0.1][0.1]
    \figByWidthb{  \width}{0}{fig/rpiEigenVector1}{\width}[0.08][0.11][0.1][0.1]
    \figByWidthb{2*\width}{0}{fig/rpiEigenVector2}{\width}[0.08][0.11][0.1][0.1]
    \figByWidthb{3*\width}{0}{fig/rpiEigenVector3}{\width}[0.08][0.11][0.1][0.1]
  \end{scope}
  \begin{scope}[yshift=2*\width]   
    \figByWidthb{       0}{0}{fig/rpiEigenVector4}{\width}[0.08][0.11][0.1][0.1]
    \figByWidthb{  \width}{0}{fig/rpiEigenVector5}{\width}[0.08][0.11][0.1][0.1]
    \figByWidthb{2*\width}{0}{fig/rpiEigenVector6}{\width}[0.08][0.11][0.1][0.1]
    \figByWidthb{3*\width}{0}{fig/rpiEigenVector7}{\width}[0.08][0.11][0.1][0.1]
  \end{scope}
  \begin{scope}[yshift=1*\width]   
    \figByWidthb{       0}{0}{fig/rpiEigenVector8}{\width}[0.08][0.11][0.1][0.1]
    \figByWidthb{  \width}{0}{fig/rpiEigenVector9}{\width}[0.08][0.11][0.1][0.1]
    \figByWidthb{2*\width}{0}{fig/rpiEigenVector10}{\width}[0.08][0.11][0.1][0.1]
    \figByWidthb{3*\width}{0}{fig/rpiEigenVector11}{\width}[0.08][0.11][0.1][0.1]
  \end{scope}
  \begin{scope}[yshift=0*\width]   
    \figByWidthb{       0}{0}{fig/rpiEigenVector12}{\width}[0.08][0.11][0.1][0.1]
    \figByWidthb{  \width}{0}{fig/rpiEigenVector13}{\width}[0.08][0.11][0.1][0.1]
    \figByWidthb{2*\width}{0}{fig/rpiEigenVector14}{\width}[0.08][0.11][0.1][0.1]
    \figByWidthb{3*\width}{0}{fig/rpiEigenVector15}{\width}[0.08][0.11][0.1][0.1]
  \end{scope}      

  % Draw grid for placing figure:
  % \draw[step=1cm,gray] (0,0) grid (16,16);
\end{tikzpicture}
\end{center}
\caption{Computed eigenfunctions for the letters RPI, arranged by magnitude of the eigenvalues.
     Outer boundary is periodic in x and Dirichlet on top/bottom, Neumann BCs on the letters.
    }
\label{fig:rpiEigenfunctions}
\end{figure}
}

% -------------------------------------------------------------------------------------------------
{
\newcommand{\width}{4cm}
\begin{figure}[htb]
\begin{center}
\begin{tikzpicture}
   \useasboundingbox (0,.5) rectangle (16,16);  % set the bounding box (so we have less surrounding white space)
  \begin{scope}[yshift=3*\width]   
    \figByWidthb{       0}{0}{fig/rpiSelectedEigenVector64}{\width}[0.08][0.11][0.1][0.1]
    \figByWidthb{  \width}{0}{fig/rpiSelectedEigenVector128}{\width}[0.08][0.11][0.1][0.1]
    \figByWidthb{2*\width}{0}{fig/rpiSelectedEigenVector192}{\width}[0.08][0.11][0.1][0.1]
    \figByWidthb{3*\width}{0}{fig/rpiSelectedEigenVector256}{\width}[0.08][0.11][0.1][0.1]
  \end{scope}
  \begin{scope}[yshift=2*\width]   
    \figByWidthb{       0}{0}{fig/rpiSelectedEigenVector320}{\width}[0.08][0.11][0.1][0.1]
    \figByWidthb{  \width}{0}{fig/rpiSelectedEigenVector384}{\width}[0.08][0.11][0.1][0.1]
    \figByWidthb{2*\width}{0}{fig/rpiSelectedEigenVector448}{\width}[0.08][0.11][0.1][0.1]
    \figByWidthb{3*\width}{0}{fig/rpiSelectedEigenVector512}{\width}[0.08][0.11][0.1][0.1]
  \end{scope}
  \begin{scope}[yshift=1*\width]   
    \figByWidthb{       0}{0}{fig/rpiSelectedEigenVector576}{\width}[0.08][0.11][0.1][0.1]
    \figByWidthb{  \width}{0}{fig/rpiSelectedEigenVector640}{\width}[0.08][0.11][0.1][0.1]
    \figByWidthb{2*\width}{0}{fig/rpiSelectedEigenVector704}{\width}[0.08][0.11][0.1][0.1]
    \figByWidthb{3*\width}{0}{fig/rpiSelectedEigenVector768}{\width}[0.08][0.11][0.1][0.1]
  \end{scope}
  \begin{scope}[yshift=0*\width]   
    \figByWidthb{       0}{0}{fig/rpiSelectedEigenVector832}{\width}[0.08][0.11][0.1][0.1]
    \figByWidthb{  \width}{0}{fig/rpiSelectedEigenVector896}{\width}[0.08][0.11][0.1][0.1]
    \figByWidthb{2*\width}{0}{fig/rpiSelectedEigenVector960}{\width}[0.08][0.11][0.1][0.1]
   %  \figByWidthb{3*\width}{0}{fig/rpiSelectedEigenVector15}{\width}[0.08][0.11][0.1][0.1]
  \end{scope}      

  % Draw grid for placing figure:
  % \draw[step=1cm,gray] (0,0) grid (16,16);
\end{tikzpicture}
\end{center}
\caption{Selected computed eigenfunctions for the letters RPI, arranged by magnitude of the eigenvalues.
     Outer boundary is periodic in x and Dirichlet on top/bottom, Neumann BCs on the letters.
     Modes 64, 128, ...
    }
\label{fig:rpiSelectedEigenfunctions}
\end{figure}
}