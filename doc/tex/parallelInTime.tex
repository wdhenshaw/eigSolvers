\section{Parallel in time solver} 

Suppose that we need to compute the solution to the IBVP quickly and we have a big parallel computer
with lots of memory and lots of processors. 

A typical finite-difference or finite element solver computes the solution to the IBVP at some $t$ 
by time-marching from time zero. The solver can be parallelized in space but there is no
obvious way to parallelize this approach in time due to the sequential nature of the time-stepping
algorithm. Thus we limited in our parallel speedup. Time-stepping thus becomes the sequential bottleneck.

The eigenfunction expansion~\eqref{eq:eigenfunctionSolution}, however, can be used to evaluate
the solution at any arbitrary time.
This expansion can be used as the basis for
a parallel in time solver. Once we know the eigenfunctions $\phi_j(\xv)$ then 
the solution at any time $t$ can be often be computed at a fixed cost, independent of the time $t$,
by summing the expansion for a finite number of terms (to the desired accuracy).

If the eigenfunctions are not known, they can be computed, in parallel.  
Different eigenfunctions can be computed independently if needed.