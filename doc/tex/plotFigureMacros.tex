
% ------------------------------------------------------------------------------------------------------------------------
% These macros need: \usepackage{xargs}% for optional args to \newcommandx
% \figByHeight{xa}{ya}{file}{height}[cxa][cxb][cya][cyb]
\newcommandx{\figByHeight}[9][5=0, 6=0, 7=0, 8=0,9=]{
\draw (#1,#2) node[anchor=south west,xshift=-16pt,yshift=-4pt] {\trimh{#3}{#4}{#5}{#6}{#7}{#8}};}
% draw bounding box: 
\newcommandx{\figByHeightb}[9][5=0, 6=0, 7=0, 8=0,9=]{
\draw (#1,#2) node[anchor=south west,xshift=-16pt,yshift=-4pt] {\trimhb{#3}{#4}{#5}{#6}{#7}{#8}};}

% \figByHeightWithLabel{xa}{ya}{file}{height}[cxa][cxb][cya][cyb][label]
\newcommandx{\figByHeightWithLabel}[9][5=0, 6=0, 7=0, 8=0,9=]{
\draw (#1,#2) node[anchor=south west,xshift=-16pt,yshift=-4pt] {\trimh{#3}{#4}{#5}{#6}{#7}{#8}} node[draw=white,fill=white,inner sep=1pt,anchor=south west] {#9};}
% draw bounding box:
\newcommandx{\figByHeightWithLabelb}[9][5=0, 6=0, 7=0, 8=0,9=]{
\draw (#1,#2) node[anchor=south west,xshift=-16pt,yshift=-4pt] {\trimhb{#3}{#4}{#5}{#6}{#7}{#8}} node[draw=white,fill=white,inner sep=1pt,anchor=south west] {#9};}


% \figByWidth{xa}{ya}{file}{height}[cxa][cxb][cya][cyb]
\newcommandx{\figByWidth}[9][5=0, 6=0, 7=0, 8=0,9=]{
\draw (#1,#2) node[anchor=south west,xshift=-16pt,yshift=-4pt] {\trimw{#3}{#4}{#5}{#6}{#7}{#8}};}
% draw bounding box: 
\newcommandx{\figByWidthb}[9][5=0, 6=0, 7=0, 8=0,9=]{
\draw (#1,#2) node[anchor=south west,xshift=-16pt,yshift=-4pt] {\trimwb{#3}{#4}{#5}{#6}{#7}{#8}};}

% \figByWidthtWithLabel{xa}{ya}{file}{height}[cxa][cxb][cya][cyb][label]
\newcommandx{\figByWidthWithLabel}[9][5=0, 6=0, 7=0, 8=0,9=]{
\draw (#1,#2) node[anchor=south west,xshift=-16pt,yshift=-4pt] {\trimw{#3}{#4}{#5}{#6}{#7}{#8}} node[draw=white,fill=white,inner sep=1pt,anchor=south west] {#9};}
% draw bounding box
\newcommandx{\figByWidthWithLabelb}[9][5=0, 6=0, 7=0, 8=0,9=]{
\draw (#1,#2) node[anchor=south west,xshift=-16pt,yshift=-4pt] {\trimwb{#3}{#4}{#5}{#6}{#7}{#8}} node[draw=white,fill=white,inner sep=1pt,anchor=south west] {#9};}
% ------------------------------------------------------------------------------------------------------------------------

%
% ============================ Plot TWO matlab figures ==============================
% USAGE: \plotTwoMatlabFigs{fig1}{fig2}{caption}{label}
% 
\newcommand{\plotTwoMatlabFigs}[4]{%
\begin{figure}[htb]
\begin{center}
\begin{tikzpicture}[scale=1]
  \useasboundingbox (0,.25) rectangle (16,6.75);  % set the bounding box (so we have less surrounding white space)

   \draw(0.0,0.0) node[anchor=south west,xshift=-15pt,yshift=-8pt] {\trimw{#1}{7.5cm}{.0}{.0}{.0}{.0}};% 7.5=figwidth 
   \draw(8.0,0.0) node[anchor=south west,xshift=-15pt,yshift=-8pt] {\trimw{#2}{7.5cm}{.0}{.0}{.0}{.0}};
  % Draw grid for placing figure:
  %\draw[step=1cm,gray] (0,0) grid (16,7);
\end{tikzpicture}
\end{center}
\caption{#3}
\label{#4}
\end{figure}
}% end plotTwoMatlabFigs

%
% ============================ Plot THREE matlab figures ==============================
% USAGE: \plotThreeMatlabFigs{fig1}{fig2}{fig3}{caption}{label}
% 
\newcommand{\plotThreeMatlabFigs}[5]{%
\begin{figure}[htb]
\begin{center}
\begin{tikzpicture}[scale=1]
  \useasboundingbox (.5,.25) rectangle (16.5,5);  % set the bounding box (so we have less surrounding white space)

   \draw(0.0, 0.0) node[anchor=south west,xshift=-15pt,yshift=-8pt] {\trimw{#1}{5.5cm}{.0}{.0}{.0}{.0}};% 5.5=figwidth 
   \draw(5.75,0.0) node[anchor=south west,xshift=-15pt,yshift=-8pt] {\trimw{#2}{5.5cm}{.0}{.0}{.0}{.0}};% 5.5=figwidth 
   \draw(11.5,0.0) node[anchor=south west,xshift=-15pt,yshift=-8pt] {\trimw{#3}{5.5cm}{.0}{.0}{.0}{.0}};
  % Draw grid for placing figure:
  % \draw[step=1cm,gray] (0.5,0) grid (16.5,5);
\end{tikzpicture}
\end{center}
\caption{#4}
\label{#5}
\end{figure}
}% end plotThreeMatlabFigs


%
% ============================ Plot ONE Figure by WIDTH ==============================
% USAGE: \plotOneFigByWidth{fig1}{caption}{label}{width}{height}
% 
\newcommand{\plotOneFigByWidth}[5]{%
\begin{figure}[htb]
\begin{center}
\begin{tikzpicture}[scale=1]
  \useasboundingbox (0,.25) rectangle (#4,#5);  % set the bounding box (so we have less surrounding white space)

   \draw(0.0,0.0) node[anchor=south west,xshift=-16pt,yshift=-5pt] {\trimw{#1}{#4}{.0}{.0}{.0}{.0}};
  % Draw grid for placing figure:
  % \draw[step=1cm,gray] (0,0) grid (#4,#5);
\end{tikzpicture}
\end{center}
\caption{#2}
\label{#3}
\end{figure}
}% end plotOneFigByWidth

%
% ============================ Plot ONE Figure by HEIGHT ==============================
% USAGE: \plotOneFigByHeight{fig1}{caption}{label}{width}{height}
% 
\newcommand{\plotOneFigByHeight}[5]{%
\begin{figure}[htb]
\begin{center}
\begin{tikzpicture}[scale=1]
  \useasboundingbox (0,.25) rectangle (#4,#5);  % set the bounding box (so we have less surrounding white space)

   \draw(0.0,0.0) node[anchor=south west,xshift=-16pt,yshift=-5pt] {\trimh{#1}{#5}{.0}{.0}{.0}{.0}};
  % Draw grid for placing figure:
  % \draw[step=1cm,gray] (0,0) grid (#4,#5);
\end{tikzpicture}
\end{center}
\caption{#2}
\label{#3}
\end{figure}
}% end plotOneFigByHeight



%
% ============================ Plot TWO Figures by Height ==============================
% USAGE: \plotTwoFigsByHeight{fig1}{fig2}{caption}{label}{height}
% 
\newcommand{\plotTwoFigsByHeight}[5]{%
\begin{figure}[htb]
\begin{center}
\begin{tikzpicture}[scale=1]
  \useasboundingbox (0,.25) rectangle (16,#5);  % set the bounding box (so we have less surrounding white space)

   \draw(0.0,0.0) node[anchor=south west,xshift=-16pt,yshift=-5pt] {\trimh{#1}{#5}{.0}{.0}{.0}{.0}};
   \draw(8.0,0.0) node[anchor=south west,xshift=-16pt,yshift=-5pt] {\trimh{#2}{#5}{.0}{.0}{.0}{.0}};
  % Draw grid for placing figure:
  % \draw[step=1cm,gray] (0,0) grid (16,#5);
\end{tikzpicture}
\end{center}
\caption{#3}
\label{#4}
\end{figure}
}% end plotTwoFigsByHeight


%
% ============================ Plot TWO Figures by WIDTH ==============================
% USAGE: \plotTwoFigsByWidth{fig1}{fig2}{caption}{label}{width}
% 
\newcommand{\plotTwoFigsByWidth}[5]{%
\begin{figure}[htb]
\begin{center}
\begin{tikzpicture}[scale=1]
  \useasboundingbox (0,.25) rectangle (16,#5);  % set the bounding box (so we have less surrounding white space)

   \draw(0.0,0.0) node[anchor=south west,xshift=-16pt,yshift=-5pt] {\trimw{#1}{#5}{.0}{.0}{.0}{.0}};
   \draw(8.0,0.0) node[anchor=south west,xshift=-16pt,yshift=-5pt] {\trimw{#2}{#5}{.0}{.0}{.0}{.0}};
  % Draw grid for placing figure:
  % \draw[step=1cm,gray] (0,0) grid (16,#5);
\end{tikzpicture}
\end{center}
\caption{#3}
\label{#4}
\end{figure}
}% end plotTwoFigsByWidth
