\subsection{Forced problem.}

\mni
{\red Note: in this section we use the new notation here: $L \phi_k = - \lambda_k^2 \phi_k $, $L=c^2\Delta$.}


Next consider the forced problem with homogeneous boundary conditions,
\bse
\bat
  &  \p_t^2 u + L u = f(\xv,t),       \quad&& \xv\in\Omega, ~ t>0 \\
  &  u(\xv,0) = u_0(\xv),   \quad&& \xv\in\Omega,       \\ 
  &  \p_t u(\xv,0) = u_1(\xv),   \quad&& \xv\in\Omega,  \\
  &  \Bc u(\xv,t) = 0         \quad&& \xv\in\p\Omega,
\eat
\ese
Let us suppose that $f(\xv,t)$ has a convergent eigenfunction expansion,
\ba
   & f(\xv) = \sum_k \fHat_kt) \phi_k(\xv) .
\ea
If $f$ is smooth and of compact support in $\Omega$ then this expansion should converge rapidly.
If $f$ is not zero at the boundary then the convergence may be slow near the boundaries (what should
we do on this case?)
As before, the solution be represented as an eigenfunction expansion,
\ba
    u(\xv,t) = \sum_{k=0}^\infty q_k(t) \phi_k(\xv),
\ea
where $q_k$ satisfies the forced ODE IVP
\ba
   & q_k'' + \lambda_k^2 q_k = \fHat_k(t), \\
   & q_k(0) = \uHat_k, \\ 
   & q_k'(0) = \vHat_k 
\ea
The general solution is {\red finish me}
\ba
  q_k(t) & = c_1 \cos(\lambda_k t) + c_2 \sin(\lambda_k t)
      + \int_0^t  \fHat_k(\tau) \, \f{\sin(\lambda_k(t-\tau))}{\lambda_k} \, d\tau, \label{eq:forcedGeneralSolution}
\ea

\mni
When $f(\xv,t)$ is non-zero near the boundary (for Dirichlet BC's) then its Fourier series will only converge in the mean.
Suppose $\fHat_k \sim 1/k$ then the forcing term in~\eqref{eq:forcedGeneralSolution} will be $O(1/k^3)$ if $\lambda_k \sim k$, I think, 
and the Fourier series for $u$ will converge reasonably fast.
Compare this to Section~\ref{sec:forcedBCII} on forced BCs which has slower convergence.



% \ba
%    q_j(t) = \uHat_j \cos(\sqrt{\lambda_j} t) + \f{1}{\sqrt{\lambda_j}} \vHat_j \sin(\sqrt{\lambda_j} t)
%        + \int_0^t ( )  ... d\tau
% \ea
Depending on the form of $\fHat_k(t)$ the particular solution can be computed analytically, otherwise
some quadrature rule can be used. These computations of $q_k$ are easy to parallelize.
