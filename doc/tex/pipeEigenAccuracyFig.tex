{
\newcommand{\figw}{6cm}
\newcommand{\figh}{6cm}

\begin{figure}[htb]
\begin{center}
\begin{tikzpicture}[scale=1]
  \useasboundingbox (0,.7) rectangle (2.2*\figw,1.05*\figh);  % set the bounding box (so we have less surrounding white space)

  \begin{scope}[xshift=0cm]
    \figByWidth{0.0}{0}{fig/pipeOrder2EigenvalueErrors}{\figw}[0][0][0][0];
  \end{scope}
  \begin{scope}[xshift=1.1*\figw]
    \figByWidth{0.0}{0}{fig/pipeOrder2EigenvectorErrors}{\figw}[0][0][0][0];
  \end{scope}  

  % Draw grid for placing figure:
  % \draw[step=1cm,gray] (0,0) grid (2.2*\figw,\figh);
\end{tikzpicture}
\end{center}
\caption{Accuracy of some computed eigenvalues and eigenvalues on a cylindrical pipe (order 2) compared to the true continuous values.
}
\label{fig:pipeAccuracyOfEigenpairsOrder2}
\end{figure}
}


{
\newcommand{\figw}{6cm}
\newcommand{\figh}{6cm}

\begin{figure}[htb]
\begin{center}
\begin{tikzpicture}[scale=1]
  \useasboundingbox (0,.7) rectangle (2.2*\figw,1.05*\figh);  % set the bounding box (so we have less surrounding white space)

  \begin{scope}[xshift=0cm]
    \figByWidth{0.0}{0}{fig/pipeOrder4EigenvalueErrors}{\figw}[0][0][0][0];
  \end{scope}
  \begin{scope}[xshift=1.1*\figw]
    \figByWidth{0.0}{0}{fig/pipeOrder4EigenvectorErrors}{\figw}[0][0][0][0];
  \end{scope}  

  % Draw grid for placing figure:
  % \draw[step=1cm,gray] (0,0) grid (2.2*\figw,\figh);
\end{tikzpicture}
\end{center}
\caption{Accuracy of some computed eigenvalues and eigenvalues on a cylindrical pipe (order 4) compared to the true continuous values.
}
\label{fig:pipeAccuracyOfEigenpairsOrder4}
\end{figure}
}

