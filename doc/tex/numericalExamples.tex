\section{Motivational examples} 

What are some good motivational examples that can make use of the method?

\begin{enumerate}
  \item Acoustics: Given a structure (or under-water vehicle,say), compute the maximum pressure
    on the structure for pressure sources located at different locations around the structure. Structure is
      fixed but solutions to many initial value problems are desired.
  \item Electromagnetics : inverse design of the dispersive properties of meta-atoms in a dispersive
      material. Geometry of meta-atoms is fixed but dispersive properties can vary. Far field response
      for a broad-band signal is desired.
  \item Elasticity: similar problem to acoustics -- find maximum stress on a elastic structure due
    to different source terms.
  \item Seismic applications : 
\end{enumerate}



% ==========================================================================================================
\clearpage
\section{Computing eigenfunctions and fitting functions}


The routine {\tt genEigs.bC} and class {\tt Ogev} can be used to compute discrete
approximations to eigenfunctions on overset grids using the {\tt SLEPc} solver.

\mni
\textbf{ Notes:}
\begin{enumerate}
  \item SLEPc does not generally return orthogonal eigenvectors
      for the eigenspace of a multiple eigenvalue. 
  \item The eigenvectors for a multiple eigenvalue can be orthogonalized (e.g. using a QR algorithm).
   However, it appears that sometimes SLEPc 
   returns essentially the same eigenvector more than once for a multiple eigenvalue. In this
   case the  orthogonalization fails -- need to fix this. May be able to provide SLEPc with
   some starting guesses. 
\end{enumerate}

% --------------------------------------------------
\clearpage
\subsection{Timings to compute eigenvalues and eigenvectors}

Here are some timings for computing the eigenvalues/eigenvectors of the RPI grid (serial, cg6). 
\begin{Verbatim}[fontsize=\footnotesize]
genEigs -noplot eigs.cmd -problem=laplace -eigCase=rpi -g=rpiGride4pn.order2.hdf -eigOption=1 
-eps_s=largest_magnitude -numEigenValues=260 -numEigenVectors=256 -tol=1.e-12 -bc5=n -bc6=n -bc7=n 

--------------- Ogev : computeEigenvalues --------------------
   numberOfGridPoints=81975, numEigenvalues=260, numEigenVectors=256
      total cpu =  1.04e+02 (s) (includes build matrix)
 cpu/eigenvalue =  4.02e-01 (s)
 Number of iterations of the method: 2
 Number of linear iterations of the method: 665
 Solution method: krylovschur


genEigs -noplot eigs.cmd -problem=laplace -eigCase=rpi -g=rpiGride4pn.order2.hdf -eigOption=1 
-eps_s=largest_magnitude -numEigenValues=130 -numEigenVectors=128 -tol=1.e-12 -bc5=n -bc6=n -bc7=n 

--------------- Ogev : computeEigenvalues --------------------
   numberOfGridPoints=81975, numEigenvalues=130, numEigenVectors=128
      total cpu =  3.62e+01 (s) (includes build matrix)
 cpu/eigenvalue =  2.78e-01 (s)
 Number of iterations of the method: 2
 Number of linear iterations of the method: 337
 Solution method: krylovschur


genEigs -noplot eigs.cmd -problem=laplace -eigCase=rpi -g=rpiGride4pn.order2.hdf -eigOption=1
 -eps_s=largest_magnitude -numEigenValues=20 -numEigenVectors=16 -tol=1.e-12 -bc5=n -bc6=n -bc7=n 

--------------- Ogev : computeEigenvalues --------------------
   numberOfGridPoints=81975, numEigenvalues=20, numEigenVectors=16
      total cpu =  1.24e+01 (s) (includes build matrix)
 cpu/eigenvalue =  6.20e-01 (s)
 Number of iterations of the method: 3
 Number of linear iterations of the method: 67
 Solution method: krylovschur


\end{Verbatim}


% --------------------------------------------------
\subsection{Computed eigenfunctions of a square}

Here are some computed eigenvalues for a square. It looks like high-order accurate schemes are
useful if a high degree of accuracy is needed. Even double eigenvalues can be computed accurately in this case.

\mni
ORDER=4
\begin{Verbatim}[fontsize=\footnotesize]
genEigs -noplot eigs.cmd -problem=laplace -g=square64.order4 -eigOption=1 -eps_s=largest_magnitude 
-numEigenValues=70  -numEigenVectors=64 -tol=1.e-12 -bc1=d -show=squareEigs64.show
Eigenvalue   0 : k=   19.7392082729 + (-0) I,  true=1.97392088021787e+01, err=5.29e-07, rel-err=2.68e-08
Eigenvalue   1 : k=   49.3480045834 + (-0) I,  true=4.93480220054468e+01, err=1.74e-05, rel-err=3.53e-07
Eigenvalue   2 : k=   49.3480045834 + (-0) I,  true=4.93480220054468e+01, err=1.74e-05, rel-err=3.53e-07
Eigenvalue   3 : k=   78.9568008939 + (-0) I,  true=7.89568352087149e+01, err=3.43e-05, rel-err=4.35e-07
Eigenvalue   4 : k=   98.6958441746 + (-0) I,  true=9.86960440108936e+01, err=2.00e-04, rel-err=2.02e-06
Eigenvalue   5 : k=   98.6958441746 + (-0) I,  true=9.86960440108936e+01, err=2.00e-04, rel-err=2.02e-06
Eigenvalue   6 : k=  128.3046404852 + (-0) I,  true=1.28304857214162e+02, err=2.17e-04, rel-err=1.69e-06
Eigenvalue   7 : k=  128.3046404852 + (-0) I,  true=1.28304857214162e+02, err=2.17e-04, rel-err=1.69e-06
Eigenvalue   8 : k=  167.7821211200 + (-0) I,  true=1.67783274818519e+02, err=1.15e-03, rel-err=6.88e-06
Eigenvalue   9 : k=  167.7821211200 + (-0) I,  true=1.67783274818519e+02, err=1.15e-03, rel-err=6.88e-06
Eigenvalue  10 : k=  177.6524800764 + (-0) I,  true=1.77652879219608e+02, err=3.99e-04, rel-err=2.25e-06
Eigenvalue  11 : k=  197.3909174305 + (-0) I,  true=1.97392088021787e+02, err=1.17e-03, rel-err=5.93e-06
Eigenvalue  12 : k=  197.3909174305 + (-0) I,  true=1.97392088021787e+02, err=1.17e-03, rel-err=5.93e-06
Eigenvalue  13 : k=  246.7387570217 + (-0) I,  true=2.46740110027234e+02, err=1.35e-03, rel-err=5.48e-06
Eigenvalue  14 : k=  246.7387570217 + (-0) I,  true=2.46740110027234e+02, err=1.35e-03, rel-err=5.48e-06
Eigenvalue  15 : k=  256.6051595574 + (-0) I,  true=2.56609714428323e+02, err=4.55e-03, rel-err=1.78e-05
\end{Verbatim}

\mni
ORDER=4 : Finer grid. Note quite 4th-order convergence. *check me*
\begin{Verbatim}[fontsize=\footnotesize]
 genEigs -noplot eigs.cmd -problem=laplace -g=square128.order4 -eigOption=1 -eps_s=largest_magnitude 
  -numEigenValues=70  -numEigenVectors=64 -tol=1.e-14 -bc1=d 

Eigenvalue   0 : k=   19.7392087459 + (-0) I,  true=1.97392088021787e+01, err=5.63e-08, rel-err=2.85e-09
Eigenvalue   1 : k=   49.3480201749 + (-0) I,  true=4.93480220054468e+01, err=1.83e-06, rel-err=3.71e-08
Eigenvalue   2 : k=   49.3480201749 + (-0) I,  true=4.93480220054468e+01, err=1.83e-06, rel-err=3.71e-08
Eigenvalue   3 : k=   78.9568316039 + (-0) I,  true=7.89568352087149e+01, err=3.60e-06, rel-err=4.57e-08
Eigenvalue   4 : k=   98.6960234232 + (-0) I,  true=9.86960440108936e+01, err=2.06e-05, rel-err=2.09e-07
Eigenvalue   5 : k=   98.6960234232 + (-0) I,  true=9.86960440108936e+01, err=2.06e-05, rel-err=2.09e-07
Eigenvalue   6 : k=  128.3048348522 + (-0) I,  true=1.28304857214162e+02, err=2.24e-05, rel-err=1.74e-07
Eigenvalue   7 : k=  128.3048348522 + (-0) I,  true=1.28304857214162e+02, err=2.24e-05, rel-err=1.74e-07
Eigenvalue   8 : k=  167.7831590477 + (-0) I,  true=1.67783274818519e+02, err=1.16e-04, rel-err=6.90e-07
Eigenvalue   9 : k=  167.7831590477 + (-0) I,  true=1.67783274818519e+02, err=1.16e-04, rel-err=6.90e-07
Eigenvalue  10 : k=  177.6528381005 + (-0) I,  true=1.77652879219608e+02, err=4.11e-05, rel-err=2.31e-07
Eigenvalue  11 : k=  197.3919704767 + (-0) I,  true=1.97392088021787e+02, err=1.18e-04, rel-err=5.95e-07
Eigenvalue  12 : k=  197.3919704767 + (-0) I,  true=1.97392088021787e+02, err=1.18e-04, rel-err=5.95e-07
Eigenvalue  13 : k=  246.7399737250 + (-0) I,  true=2.46740110027234e+02, err=1.36e-04, rel-err=5.52e-07
Eigenvalue  14 : k=  246.7399737250 + (-0) I,  true=2.46740110027234e+02, err=1.36e-04, rel-err=5.52e-07
Eigenvalue  15 : k=  256.6092717758 + (-0) I,  true=2.56609714428323e+02, err=4.43e-04, rel-err=1.73e-06

\end{Verbatim}

% Finer grid. 
% \begin{Verbatim}[fontsize=\footnotesize]
% genEigs -noplot eigs.cmd -problem=laplace -g=square256.order4 -eigOption=1 -eps_s=largest_magnitude
%  -numEigenValues=70 -numEigenVectors=64 -tol=1.e-14 -bc1=d -show=squareEigs64.show
% Eigenvalue   0 : k=   19.7392087980 + (-0) I,  true=1.97392088021787e+01, err=4.22e-09, rel-err=2.14e-10
% Eigenvalue   1 : k=   49.3480218675 + (-0) I,  true=4.93480220054468e+01, err=1.38e-07, rel-err=2.80e-09
% Eigenvalue   2 : k=   49.3480218675 + (-0) I,  true=4.93480220054468e+01, err=1.38e-07, rel-err=2.80e-09
% Eigenvalue   3 : k=   78.9568349370 + (-0) I,  true=7.89568352087149e+01, err=2.72e-07, rel-err=3.44e-09
% Eigenvalue   4 : k=   98.6960424611 + (-0) I,  true=9.86960440108936e+01, err=1.55e-06, rel-err=1.57e-08
% Eigenvalue   5 : k=   98.6960424611 + (-0) I,  true=9.86960440108936e+01, err=1.55e-06, rel-err=1.57e-08
% \end{Verbatim}

\mni 
ORDER=8 -- this seems like a good way to get accurate eigenvalues
\begin{Verbatim}[fontsize=\footnotesize]
genEigs -noplot eigs.cmd -problem=laplace -g=square128.order8 -eigOption=1 -eps_s=largest_magnitude 
  -numEigenValues=70  -numEigenVectors=64 -tol=1.e-14 -bc1=d 

Eigenvalue   0 : k=   19.7392088022 + (-0) I,  true=1.97392088021787e+01, err=9.44e-12, rel-err=4.78e-13
Eigenvalue   1 : k=   49.3480220055 + (-0) I,  true=4.93480220054468e+01, err=6.97e-12, rel-err=1.41e-13
Eigenvalue   2 : k=   49.3480220055 + (-0) I,  true=4.93480220054468e+01, err=1.10e-11, rel-err=2.22e-13
Eigenvalue   3 : k=   78.9568352087 + (-0) I,  true=7.89568352087149e+01, err=1.43e-11, rel-err=1.81e-13
Eigenvalue   4 : k=   98.6960440110 + (-0) I,  true=9.86960440108936e+01, err=1.44e-10, rel-err=1.46e-12
Eigenvalue   5 : k=   98.6960440110 + (-0) I,  true=9.86960440108936e+01, err=1.55e-10, rel-err=1.57e-12
Eigenvalue   6 : k=  128.3048572143 + (-0) I,  true=1.28304857214162e+02, err=1.49e-10, rel-err=1.16e-12
Eigenvalue   7 : k=  128.3048572143 + (-0) I,  true=1.28304857214162e+02, err=1.54e-10, rel-err=1.20e-12
Eigenvalue   8 : k=  167.7832748210 + (-0) I,  true=1.67783274818519e+02, err=2.45e-09, rel-err=1.46e-11
Eigenvalue   9 : k=  167.7832748210 + (-0) I,  true=1.67783274818519e+02, err=2.45e-09, rel-err=1.46e-11
Eigenvalue  10 : k=  177.6528792199 + (-0) I,  true=1.77652879219608e+02, err=2.94e-10, rel-err=1.66e-12
Eigenvalue  11 : k=  197.3920880242 + (-0) I,  true=1.97392088021787e+02, err=2.45e-09, rel-err=1.24e-11
Eigenvalue  12 : k=  197.3920880242 + (-0) I,  true=1.97392088021787e+02, err=2.45e-09, rel-err=1.24e-11

\end{Verbatim}



Figure~\ref{} shows some computed eigenfunctions of a square.


% -------------------------------------------------------------------------------------------
\subsection{Computed eigenfunctions of a disk}


There are many multiplicity-two eigenvalues for the disk dues to the angular symmetry; 
these correspond to eigenfunctions
of the forms
\ba
     \phi_1(r,\theta) &= J_n(\kappa r) \cos(n\theta) , \\
     \phi_2(r,\theta) &= J_n(\kappa r) \sin(n\theta) ,
\ea


Here are some computed eigenvalues for a disk.
\begin{Verbatim}[fontsize=\footnotesize]
genEigs -noplot eigs.cmd -problem=laplace -eigCase=disk -g=sice4.order4 -eigOption=1 -eps_s=largest_magnitude
 -numEigenValues=40 -numEigenVectors=32 -tol=1.e-12 -bc1=d
Eigenvalue   0 : k=    5.7831856376 + (-0) I,  true=5.78318596294677e+00, err=3.25e-07, rel-err=5.63e-08
Eigenvalue   1 : k=   14.6819663423 + (-0) I,  true=1.46819706421239e+01, err=4.30e-06, rel-err=2.93e-07
Eigenvalue   2 : k=   14.6819663423 + (-0) I,  true=1.46819706421239e+01, err=4.30e-06, rel-err=2.93e-07
Eigenvalue   3 : k=   26.3745765958 + (-0) I,  true=2.63746164271634e+01, err=3.98e-05, rel-err=1.51e-06
Eigenvalue   4 : k=   26.3746064090 + (-0) I,  true=2.63746164271634e+01, err=1.00e-05, rel-err=3.80e-07
Eigenvalue   5 : k=   30.4712325478 + (-0) I,  true=3.04712623436621e+01, err=2.98e-05, rel-err=9.78e-07
Eigenvalue   6 : k=   40.7063642227 + (-0) I,  true=4.07064658182003e+01, err=1.02e-04, rel-err=2.50e-06
Eigenvalue   7 : k=   40.7063642227 + (-0) I,  true=4.07064658182003e+01, err=1.02e-04, rel-err=2.50e-06
Eigenvalue   8 : k=   49.2183375312 + (-0) I,  true=4.92184563216946e+01, err=1.19e-04, rel-err=2.41e-06
\end{Verbatim}

Figure~\ref{fig:diskEigenfunctions} shows some computed eigenfunctions of a disk, arranged by magnitude of the
  eigenvalues.

% -------------------------------------------------------------------------------------------------
{
\newcommand{\width}{4cm}
\begin{figure}[htb]
\begin{center}
\begin{tikzpicture}
   \useasboundingbox (0,.5) rectangle (16,16);  % set the bounding box (so we have less surrounding white space)
  \begin{scope}[yshift=3*\width]   
    \figByWidthb{       0}{0}{fig/diskEigenVector0}{\width}[0.08][0.11][0.1][0.1]
    \figByWidthb{  \width}{0}{fig/diskEigenVector1}{\width}[0.08][0.11][0.1][0.1]
    \figByWidthb{2*\width}{0}{fig/diskEigenVector2}{\width}[0.08][0.11][0.1][0.1]
    \figByWidthb{3*\width}{0}{fig/diskEigenVector3}{\width}[0.08][0.11][0.1][0.1]
  \end{scope}
  \begin{scope}[yshift=2*\width]   
    \figByWidthb{       0}{0}{fig/diskEigenVector4}{\width}[0.08][0.11][0.1][0.1]
    \figByWidthb{  \width}{0}{fig/diskEigenVector5}{\width}[0.08][0.11][0.1][0.1]
    \figByWidthb{2*\width}{0}{fig/diskEigenVector6}{\width}[0.08][0.11][0.1][0.1]
    \figByWidthb{3*\width}{0}{fig/diskEigenVector7}{\width}[0.08][0.11][0.1][0.1]
  \end{scope}
  \begin{scope}[yshift=1*\width]   
    \figByWidthb{       0}{0}{fig/diskEigenVector8}{\width}[0.08][0.11][0.1][0.1]
    \figByWidthb{  \width}{0}{fig/diskEigenVector9}{\width}[0.08][0.11][0.1][0.1]
    \figByWidthb{2*\width}{0}{fig/diskEigenVector10}{\width}[0.08][0.11][0.1][0.1]
    \figByWidthb{3*\width}{0}{fig/diskEigenVector11}{\width}[0.08][0.11][0.1][0.1]
  \end{scope}
  \begin{scope}[yshift=0*\width]   
    \figByWidthb{       0}{0}{fig/diskEigenVector12}{\width}[0.08][0.11][0.1][0.1]
    \figByWidthb{  \width}{0}{fig/diskEigenVector13}{\width}[0.08][0.11][0.1][0.1]
    \figByWidthb{2*\width}{0}{fig/diskEigenVector14}{\width}[0.08][0.11][0.1][0.1]
    \figByWidthb{3*\width}{0}{fig/diskEigenVector15}{\width}[0.08][0.11][0.1][0.1]
  \end{scope}      

  % Draw grid for placing figure:
  % \draw[step=1cm,gray] (0,0) grid (16,16);
\end{tikzpicture}
\end{center}
\caption{Computed eigenfunctions of a disk, arranged by magnitude of the eigenvalues.}
\label{fig:diskEigenfunctions}
\end{figure}
}
% -------------------------------------------------------------------------------------------------


% -------------------------------------------------------------------------------------------
\clearpage
\subsection{Computed eigenfunctions of an ellipse in a box}

Here are some computed eigenvalues for the ellipse in a box. There seem to be no multiple eigenvalues.
\begin{Verbatim}[fontsize=\footnotesize]
genEigs -noplot eigs.cmd -problem=laplace -eigCase=disk -g=ellipsee4.order4.s1.5.hdf -eigOption=1 
  -eps_s=largest_magnitude -numEigenValues=68 -numEigenVectors=64 -tol=1.e-6 -bc1=d -show=eib64.show

Eigenvalue   0 : k=    3.3579349838 + (-0) I,  
Eigenvalue   1 : k=    3.7172899474 + (-0) I,  
Eigenvalue   2 : k=    4.2746640412 + (-0) I,  
Eigenvalue   3 : k=    5.1551742507 + (-0) I,  
Eigenvalue   4 : k=    6.7306843897 + (-0) I,  
Eigenvalue   5 : k=    8.3202600219 + (-0) I,  
Eigenvalue   6 : k=    8.8427800346 + (-0) I,  
Eigenvalue   7 : k=   10.3307519679 + (-0) I,  
Eigenvalue   8 : k=   12.3634449757 + (-0) I,  
Eigenvalue   9 : k=   12.5520770161 + (-0) I,  
Eigenvalue  10 : k=   12.9907804080 + (-0) I,  
Eigenvalue  11 : k=   13.4171548012 + (-0) I,  
Eigenvalue  12 : k=   14.6996193850 + (-0) I,  
Eigenvalue  13 : k=   16.9199964707 + (-0) I,  
Eigenvalue  14 : k=   17.3127037296 + (-0) I,  
Eigenvalue  15 : k=   18.1024666484 + (-0) I,  
Eigenvalue  16 : k=   18.5014568730 + (-0) I,  
Eigenvalue  17 : k=   20.6358025973 + (-0) I,  
Eigenvalue  18 : k=   21.0111189039 + (-0) I,  
Eigenvalue  19 : k=   21.4286701882 + (-0) I,  
Eigenvalue  20 : k=   23.2851814358 + (-0) I,  
Eigenvalue  21 : k=   24.7564697513 + (-0) I,  
Eigenvalue  22 : k=   24.9029841436 + (-0) I,  

\end{Verbatim}


% -------------------------------------------------------------------------------------------------
{
\newcommand{\width}{4cm}
\begin{figure}[htb]
\begin{center}
\begin{tikzpicture}
   \useasboundingbox (0,.5) rectangle (16,16);  % set the bounding box (so we have less surrounding white space)
  \begin{scope}[yshift=3*\width]   
    \figByWidthb{       0}{0}{fig/ellipseEigenVector0}{\width}[0.08][0.11][0.1][0.1]
    \figByWidthb{  \width}{0}{fig/ellipseEigenVector1}{\width}[0.08][0.11][0.1][0.1]
    \figByWidthb{2*\width}{0}{fig/ellipseEigenVector2}{\width}[0.08][0.11][0.1][0.1]
    \figByWidthb{3*\width}{0}{fig/ellipseEigenVector3}{\width}[0.08][0.11][0.1][0.1]
  \end{scope}
  \begin{scope}[yshift=2*\width]   
    \figByWidthb{       0}{0}{fig/ellipseEigenVector4}{\width}[0.08][0.11][0.1][0.1]
    \figByWidthb{  \width}{0}{fig/ellipseEigenVector5}{\width}[0.08][0.11][0.1][0.1]
    \figByWidthb{2*\width}{0}{fig/ellipseEigenVector6}{\width}[0.08][0.11][0.1][0.1]
    \figByWidthb{3*\width}{0}{fig/ellipseEigenVector7}{\width}[0.08][0.11][0.1][0.1]
  \end{scope}
  \begin{scope}[yshift=1*\width]   
    \figByWidthb{       0}{0}{fig/ellipseEigenVector8}{\width}[0.08][0.11][0.1][0.1]
    \figByWidthb{  \width}{0}{fig/ellipseEigenVector9}{\width}[0.08][0.11][0.1][0.1]
    \figByWidthb{2*\width}{0}{fig/ellipseEigenVector10}{\width}[0.08][0.11][0.1][0.1]
    \figByWidthb{3*\width}{0}{fig/ellipseEigenVector11}{\width}[0.08][0.11][0.1][0.1]
  \end{scope}
  \begin{scope}[yshift=0*\width]   
    \figByWidthb{       0}{0}{fig/ellipseEigenVector12}{\width}[0.08][0.11][0.1][0.1]
    \figByWidthb{  \width}{0}{fig/ellipseEigenVector13}{\width}[0.08][0.11][0.1][0.1]
    \figByWidthb{2*\width}{0}{fig/ellipseEigenVector14}{\width}[0.08][0.11][0.1][0.1]
    \figByWidthb{3*\width}{0}{fig/ellipseEigenVector15}{\width}[0.08][0.11][0.1][0.1]
  \end{scope}      

  % Draw grid for placing figure:
  % \draw[step=1cm,gray] (0,0) grid (16,16);
\end{tikzpicture}
\end{center}
\caption{Computed eigenfunctions of an ellipse in a box, arranged by magnitude of the eigenvalues.}
\label{fig:ellipseInABoxEigenfunctions}
\end{figure}
}



% -------------------------------------------------------------------------------------------
\clearpage
\subsection{Computed eigenfunctions for some shapes in a box}

% -------------------------------------------------------------------------------------------------
{
\newcommand{\width}{4cm}
\begin{figure}[htb]
\begin{center}
\begin{tikzpicture}
   \useasboundingbox (0,.5) rectangle (16,16);  % set the bounding box (so we have less surrounding white space)
  \begin{scope}[yshift=3*\width]   
    \figByWidthb{       0}{0}{fig/shapesEigenVector0}{\width}[0.08][0.11][0.1][0.1]
    \figByWidthb{  \width}{0}{fig/shapesEigenVector1}{\width}[0.08][0.11][0.1][0.1]
    \figByWidthb{2*\width}{0}{fig/shapesEigenVector2}{\width}[0.08][0.11][0.1][0.1]
    \figByWidthb{3*\width}{0}{fig/shapesEigenVector3}{\width}[0.08][0.11][0.1][0.1]
  \end{scope}
  \begin{scope}[yshift=2*\width]   
    \figByWidthb{       0}{0}{fig/shapesEigenVector4}{\width}[0.08][0.11][0.1][0.1]
    \figByWidthb{  \width}{0}{fig/shapesEigenVector5}{\width}[0.08][0.11][0.1][0.1]
    \figByWidthb{2*\width}{0}{fig/shapesEigenVector6}{\width}[0.08][0.11][0.1][0.1]
    \figByWidthb{3*\width}{0}{fig/shapesEigenVector7}{\width}[0.08][0.11][0.1][0.1]
  \end{scope}
  \begin{scope}[yshift=1*\width]   
    \figByWidthb{       0}{0}{fig/shapesEigenVector8}{\width}[0.08][0.11][0.1][0.1]
    \figByWidthb{  \width}{0}{fig/shapesEigenVector9}{\width}[0.08][0.11][0.1][0.1]
    \figByWidthb{2*\width}{0}{fig/shapesEigenVector10}{\width}[0.08][0.11][0.1][0.1]
    \figByWidthb{3*\width}{0}{fig/shapesEigenVector11}{\width}[0.08][0.11][0.1][0.1]
  \end{scope}
  \begin{scope}[yshift=0*\width]   
    \figByWidthb{       0}{0}{fig/shapesEigenVector12}{\width}[0.08][0.11][0.1][0.1]
    \figByWidthb{  \width}{0}{fig/shapesEigenVector13}{\width}[0.08][0.11][0.1][0.1]
    \figByWidthb{2*\width}{0}{fig/shapesEigenVector14}{\width}[0.08][0.11][0.1][0.1]
    \figByWidthb{3*\width}{0}{fig/shapesEigenVector15}{\width}[0.08][0.11][0.1][0.1]
  \end{scope}      

  % Draw grid for placing figure:
  % \draw[step=1cm,gray] (0,0) grid (16,16);
\end{tikzpicture}
\end{center}
\caption{Computed eigenfunctions of some shapes in a box, arranged by magnitude of the eigenvalues.}
\label{fig:shapesInABoxEigenfunctions}
\end{figure}
}
% -------------------------------------------------------------------------------------------------------


% -------------------------------------------------------------------------------------------
\clearpage
\subsection{Scattering from a cylinder}

We consider scattering of a Gaussian plane wave from a cylinder (periodic in y).
Figure~\ref{fig:cicEigenfunctions} shows some eigenfunctions.



% -------------------------------------------------------------------------------------------------
{
\newcommand{\width}{4cm}
\begin{figure}[htb]
\begin{center}
\begin{tikzpicture}
   \useasboundingbox (0,.5) rectangle (16,16);  % set the bounding box (so we have less surrounding white space)
  \begin{scope}[yshift=3*\width]   
    \figByWidthb{       0}{0}{fig/cicEigenVector0}{\width}[0.08][0.11][0.1][0.1]
    \figByWidthb{  \width}{0}{fig/cicEigenVector1}{\width}[0.08][0.11][0.1][0.1]
    \figByWidthb{2*\width}{0}{fig/cicEigenVector2}{\width}[0.08][0.11][0.1][0.1]
    \figByWidthb{3*\width}{0}{fig/cicEigenVector3}{\width}[0.08][0.11][0.1][0.1]
  \end{scope}
  \begin{scope}[yshift=2*\width]   
    \figByWidthb{       0}{0}{fig/cicEigenVector4}{\width}[0.08][0.11][0.1][0.1]
    \figByWidthb{  \width}{0}{fig/cicEigenVector5}{\width}[0.08][0.11][0.1][0.1]
    \figByWidthb{2*\width}{0}{fig/cicEigenVector6}{\width}[0.08][0.11][0.1][0.1]
    \figByWidthb{3*\width}{0}{fig/cicEigenVector7}{\width}[0.08][0.11][0.1][0.1]
  \end{scope}
  \begin{scope}[yshift=1*\width]   
    \figByWidthb{       0}{0}{fig/cicEigenVector8}{\width}[0.08][0.11][0.1][0.1]
    \figByWidthb{  \width}{0}{fig/cicEigenVector9}{\width}[0.08][0.11][0.1][0.1]
    \figByWidthb{2*\width}{0}{fig/cicEigenVector10}{\width}[0.08][0.11][0.1][0.1]
    \figByWidthb{3*\width}{0}{fig/cicEigenVector11}{\width}[0.08][0.11][0.1][0.1]
  \end{scope}
  \begin{scope}[yshift=0*\width]   
    \figByWidthb{       0}{0}{fig/cicEigenVector12}{\width}[0.08][0.11][0.1][0.1]
    \figByWidthb{  \width}{0}{fig/cicEigenVector13}{\width}[0.08][0.11][0.1][0.1]
    \figByWidthb{2*\width}{0}{fig/cicEigenVector14}{\width}[0.08][0.11][0.1][0.1]
    \figByWidthb{3*\width}{0}{fig/cicEigenVector15}{\width}[0.08][0.11][0.1][0.1]
  \end{scope}      

  % Draw grid for placing figure:
  % \draw[step=1cm,gray] (0,0) grid (16,16);
\end{tikzpicture}
\end{center}
\caption{Computed eigenfunctions of a cylinder in a box, arranged by magnitude of the eigenvalues.
    }
\label{fig:cicEigenfunctions}
\end{figure}
}
% -------------------------------------------------------------------------------------------------------

\clearpage
Here are commands to compute the scattering of a Gaussian plane wave from the cylinder.
\begin{Verbatim}[fontsize=\footnotesize]
eveSolver square.cmd -eigFile=cicEigs1024.show -beta=40 -x0=-1.25 -y0=0.5 -k0=0 -ic=gpw
Errors in u0 fit: max-err=1.28e-03, l2-err=3.78e-04
Errors in u1 fit: max-err=4.13e-02, l2-err=1.28e-02
\end{Verbatim}

2048 modes: 
\begin{Verbatim}[fontsize=\footnotesize]
eveSolver square.cmd -eigFile=cicEigs2048.show -beta=40 -x0=-1.25 -y0=0.5 -k0=0 -ic=gpw
Errors in u0 fit: max-err=2.68e-04, l2-err=3.79e-05
Errors in u1 fit: max-err=3.97e-03, l2-err=7.25e-04
\end{Verbatim}

% -------------------------------------------------------------------------------------------------
{
\newcommand{\width}{8cm}
\begin{figure}[htb]
\begin{center}
\begin{tikzpicture}
   \useasboundingbox (0,.5) rectangle (16,8);  % set the bounding box (so we have less surrounding white space)

     \figByWidthb{0*\width}{0}{fig/cicScatteringt1p5}{\width}[0.1][0.0][0.1][0.1]   
     \figByWidthb{1*\width}{0}{fig/cicScat2048Modest1p5}{\width}[0.1][0.0][0.1][0.1]   

  % Draw grid for placing figure:
  % \draw[step=1cm,gray] (0,0) grid (16,16);
\end{tikzpicture}
\end{center}
\caption{Scattering from a cylinder using 1024 eigenfunctions, $t=1.5$. Left: 1024 modes, right 2048 modes.}
\label{fig:cicScattering}
\end{figure}
}
% -------------------------------------------------------------------------------------------------------


% -------------------------------------------------------------------------------------------------
{
\newcommand{\width}{8cm}
\begin{figure}[htb]
\begin{center}
\begin{tikzpicture}
   \useasboundingbox (0,.5) rectangle (16,8);  % set the bounding box (so we have less surrounding white space)

     \figByWidthb{0*\width}{0}{fig/cicScatteringNeumann1024}{\width}[0.1][0.0][0.1][0.1]   
  %   \figByWidthb{1*\width}{0}{fig/cicScat2048Modest1p5}{\width}[0.1][0.0][0.1][0.1]   

  % Draw grid for placing figure:
  % \draw[step=1cm,gray] (0,0) grid (16,16);
\end{tikzpicture}
\end{center}
\caption{Neumann BC: scattering from a cylinder, $t=1.5$. Left: 1024 modes, right 2048 modes.}
\label{fig:cicScatteringNeumann}
\end{figure}
}
% -------------------------------------------------------------------------------------------------------



% -------------------------------------------------------------------------------------------
\clearpage
\subsection{RPI Scattering}

We consider scattering of a Gaussian plane wave from three bodies with shapes of the letters R, P and I.
Figure~\ref{fig:rpiEigenfunctions} shows some eigenfunctions.

% -------------------------------------------------------------------------------------------------
{
\newcommand{\width}{8cm}
\begin{figure}[htb]
\begin{center}
\begin{tikzpicture}
   \useasboundingbox (0,.5) rectangle (16,8);  % set the bounding box (so we have less surrounding white space)

     \figByWidthb{0*\width}{0}{fig/rpiScat1024t1p5}{\width}[0.1][0.0][0.1][0.1]   
     \figByWidthb{1*\width}{0}{fig/rpiScat1024t2p0}{\width}[0.1][0.0][0.1][0.1]   

  % Draw grid for placing figure:
  % \draw[step=1cm,gray] (0,0) grid (16,16);
\end{tikzpicture}
\end{center}
\caption{Scattering from the letters RPI, $t=1.5$, 1024 modes.
    Neumann BCs on letters. Gaussian plane wave travels upward from the bottom.}
\label{fig:rpiScatteringNeumann}
\end{figure}
}
% -------------------------------------------------------------------------------------------------------

% -------------------------------------------------------------------------------------------------
{
\newcommand{\width}{8cm}
\begin{figure}[htb]
\begin{center}
\begin{tikzpicture}
   \useasboundingbox (0,.5) rectangle (16,8);  % set the bounding box (so we have less surrounding white space)

     \figByWidthb{0*\width}{0}{fig/rpiScat4096t1p5}{\width}[0.1][0.0][0.1][0.1]   
     \figByWidthb{1*\width}{0}{fig/rpiScat4096t2p0}{\width}[0.1][0.0][0.1][0.1]   

  % Draw grid for placing figure:
  % \draw[step=1cm,gray] (0,0) grid (16,16);
\end{tikzpicture}
\end{center}
\caption{Scattering from the letters RPI, $t=1.5$, 4096 modes.
    Neumann BCs on letters. Modulated Gaussian plane wave travels upward from the bottom.}
\label{fig:rpiScattering4096Neumann}
\end{figure}
}
% -------------------------------------------------------------------------------------------------------



% -------------------------------------------------------------------------------------------------
{
\newcommand{\width}{4cm}
\begin{figure}[htb]
\begin{center}
\begin{tikzpicture}
   \useasboundingbox (0,.5) rectangle (16,16);  % set the bounding box (so we have less surrounding white space)
  \begin{scope}[yshift=3*\width]   
    \figByWidthb{       0}{0}{fig/rpiEigenVector0}{\width}[0.08][0.11][0.1][0.1]
    \figByWidthb{  \width}{0}{fig/rpiEigenVector1}{\width}[0.08][0.11][0.1][0.1]
    \figByWidthb{2*\width}{0}{fig/rpiEigenVector2}{\width}[0.08][0.11][0.1][0.1]
    \figByWidthb{3*\width}{0}{fig/rpiEigenVector3}{\width}[0.08][0.11][0.1][0.1]
  \end{scope}
  \begin{scope}[yshift=2*\width]   
    \figByWidthb{       0}{0}{fig/rpiEigenVector4}{\width}[0.08][0.11][0.1][0.1]
    \figByWidthb{  \width}{0}{fig/rpiEigenVector5}{\width}[0.08][0.11][0.1][0.1]
    \figByWidthb{2*\width}{0}{fig/rpiEigenVector6}{\width}[0.08][0.11][0.1][0.1]
    \figByWidthb{3*\width}{0}{fig/rpiEigenVector7}{\width}[0.08][0.11][0.1][0.1]
  \end{scope}
  \begin{scope}[yshift=1*\width]   
    \figByWidthb{       0}{0}{fig/rpiEigenVector8}{\width}[0.08][0.11][0.1][0.1]
    \figByWidthb{  \width}{0}{fig/rpiEigenVector9}{\width}[0.08][0.11][0.1][0.1]
    \figByWidthb{2*\width}{0}{fig/rpiEigenVector10}{\width}[0.08][0.11][0.1][0.1]
    \figByWidthb{3*\width}{0}{fig/rpiEigenVector11}{\width}[0.08][0.11][0.1][0.1]
  \end{scope}
  \begin{scope}[yshift=0*\width]   
    \figByWidthb{       0}{0}{fig/rpiEigenVector12}{\width}[0.08][0.11][0.1][0.1]
    \figByWidthb{  \width}{0}{fig/rpiEigenVector13}{\width}[0.08][0.11][0.1][0.1]
    \figByWidthb{2*\width}{0}{fig/rpiEigenVector14}{\width}[0.08][0.11][0.1][0.1]
    \figByWidthb{3*\width}{0}{fig/rpiEigenVector15}{\width}[0.08][0.11][0.1][0.1]
  \end{scope}      

  % Draw grid for placing figure:
  % \draw[step=1cm,gray] (0,0) grid (16,16);
\end{tikzpicture}
\end{center}
\caption{Computed eigenfunctions for the letters RPI, arranged by magnitude of the eigenvalues.
     Outer boundary is periodic in x and Dirichlet on top/bottom, Neumann BCs on the letters.
    }
\label{fig:rpiEigenfunctions}
\end{figure}
}
% -------------------------------------------------------------------------------------------------------

% -------------------------------------------------------------------------------------------------
{
\newcommand{\width}{4cm}
\begin{figure}[htb]
\begin{center}
\begin{tikzpicture}
   \useasboundingbox (0,.5) rectangle (16,16);  % set the bounding box (so we have less surrounding white space)
  \begin{scope}[yshift=3*\width]   
    \figByWidthb{       0}{0}{fig/rpiSelectedEigenVector64}{\width}[0.08][0.11][0.1][0.1]
    \figByWidthb{  \width}{0}{fig/rpiSelectedEigenVector128}{\width}[0.08][0.11][0.1][0.1]
    \figByWidthb{2*\width}{0}{fig/rpiSelectedEigenVector192}{\width}[0.08][0.11][0.1][0.1]
    \figByWidthb{3*\width}{0}{fig/rpiSelectedEigenVector256}{\width}[0.08][0.11][0.1][0.1]
  \end{scope}
  \begin{scope}[yshift=2*\width]   
    \figByWidthb{       0}{0}{fig/rpiSelectedEigenVector320}{\width}[0.08][0.11][0.1][0.1]
    \figByWidthb{  \width}{0}{fig/rpiSelectedEigenVector384}{\width}[0.08][0.11][0.1][0.1]
    \figByWidthb{2*\width}{0}{fig/rpiSelectedEigenVector448}{\width}[0.08][0.11][0.1][0.1]
    \figByWidthb{3*\width}{0}{fig/rpiSelectedEigenVector512}{\width}[0.08][0.11][0.1][0.1]
  \end{scope}
  \begin{scope}[yshift=1*\width]   
    \figByWidthb{       0}{0}{fig/rpiSelectedEigenVector576}{\width}[0.08][0.11][0.1][0.1]
    \figByWidthb{  \width}{0}{fig/rpiSelectedEigenVector640}{\width}[0.08][0.11][0.1][0.1]
    \figByWidthb{2*\width}{0}{fig/rpiSelectedEigenVector704}{\width}[0.08][0.11][0.1][0.1]
    \figByWidthb{3*\width}{0}{fig/rpiSelectedEigenVector768}{\width}[0.08][0.11][0.1][0.1]
  \end{scope}
  \begin{scope}[yshift=0*\width]   
    \figByWidthb{       0}{0}{fig/rpiSelectedEigenVector832}{\width}[0.08][0.11][0.1][0.1]
    \figByWidthb{  \width}{0}{fig/rpiSelectedEigenVector896}{\width}[0.08][0.11][0.1][0.1]
    \figByWidthb{2*\width}{0}{fig/rpiSelectedEigenVector960}{\width}[0.08][0.11][0.1][0.1]
   %  \figByWidthb{3*\width}{0}{fig/rpiSelectedEigenVector15}{\width}[0.08][0.11][0.1][0.1]
  \end{scope}      

  % Draw grid for placing figure:
  % \draw[step=1cm,gray] (0,0) grid (16,16);
\end{tikzpicture}
\end{center}
\caption{Selected computed eigenfunctions for the letters RPI, arranged by magnitude of the eigenvalues.
     Outer boundary is periodic in x and Dirichlet on top/bottom, Neumann BCs on the letters.
     Modes 64, 128, ...
    }
\label{fig:rpiSelectedEigenfunctions}
\end{figure}
}
% -------------------------------------------------------------------------------------------------------
