% -------------------------------------------------------------------------------------------------
{
\newcommand{\width}{4cm}
\begin{figure}[htb]
\begin{center}
\begin{tikzpicture}
   \useasboundingbox (0,.5) rectangle (16,16);  % set the bounding box (so we have less surrounding white space)
  \begin{scope}[yshift=3*\width]   
    \figByWidthb{       0}{0}{fig/ellipseEigenVector0}{\width}[0.08][0.11][0.1][0.1]
    \figByWidthb{  \width}{0}{fig/ellipseEigenVector1}{\width}[0.08][0.11][0.1][0.1]
    \figByWidthb{2*\width}{0}{fig/ellipseEigenVector2}{\width}[0.08][0.11][0.1][0.1]
    \figByWidthb{3*\width}{0}{fig/ellipseEigenVector3}{\width}[0.08][0.11][0.1][0.1]
  \end{scope}
  \begin{scope}[yshift=2*\width]   
    \figByWidthb{       0}{0}{fig/ellipseEigenVector4}{\width}[0.08][0.11][0.1][0.1]
    \figByWidthb{  \width}{0}{fig/ellipseEigenVector5}{\width}[0.08][0.11][0.1][0.1]
    \figByWidthb{2*\width}{0}{fig/ellipseEigenVector6}{\width}[0.08][0.11][0.1][0.1]
    \figByWidthb{3*\width}{0}{fig/ellipseEigenVector7}{\width}[0.08][0.11][0.1][0.1]
  \end{scope}
  \begin{scope}[yshift=1*\width]   
    \figByWidthb{       0}{0}{fig/ellipseEigenVector8}{\width}[0.08][0.11][0.1][0.1]
    \figByWidthb{  \width}{0}{fig/ellipseEigenVector9}{\width}[0.08][0.11][0.1][0.1]
    \figByWidthb{2*\width}{0}{fig/ellipseEigenVector10}{\width}[0.08][0.11][0.1][0.1]
    \figByWidthb{3*\width}{0}{fig/ellipseEigenVector11}{\width}[0.08][0.11][0.1][0.1]
  \end{scope}
  \begin{scope}[yshift=0*\width]   
    \figByWidthb{       0}{0}{fig/ellipseEigenVector12}{\width}[0.08][0.11][0.1][0.1]
    \figByWidthb{  \width}{0}{fig/ellipseEigenVector13}{\width}[0.08][0.11][0.1][0.1]
    \figByWidthb{2*\width}{0}{fig/ellipseEigenVector14}{\width}[0.08][0.11][0.1][0.1]
    \figByWidthb{3*\width}{0}{fig/ellipseEigenVector15}{\width}[0.08][0.11][0.1][0.1]
  \end{scope}      

  % Draw grid for placing figure:
  % \draw[step=1cm,gray] (0,0) grid (16,16);
\end{tikzpicture}
\end{center}
\caption{Computed eigenfunctions of an ellipse in a box, arranged by magnitude of the eigenvalues.}
\label{fig:ellipseInABoxEigenfunctions}
\end{figure}
}