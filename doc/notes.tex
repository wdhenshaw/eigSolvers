\documentclass[preprint,11pt]{elsarticle}
% \documentclass[3p,twocolumn]{elsarticle}
\usepackage[bookmarks=true,colorlinks=true,linkcolor=blue]{hyperref}

\usepackage{amsmath,amssymb,mathabx}
\usepackage{graphicx,esint,ulem}

\biboptions{sort&compress}% compress bibtex entries
\usepackage[usenames,dvipsnames,svgnames,table]{xcolor}

\usepackage{color}
\definecolor{jwbGreen}{rgb}{0, .6, 0}
\definecolor{jbaPurple}{HTML}{6600FF}
\definecolor{purple}{rgb}{.7, 0., .8}
\newcommand{\red}{\color{red}}
\newcommand{\green}{\color{green}}
\newcommand{\blue}{\color{blue}}
\newcommand{\purple}{\color{purple}}
\newcommand{\orange}{\color{orange}}

\usepackage{calc}
\usepackage[margin=1.in]{geometry}

% for fancy coloured boxes: 
\usepackage{empheq}
\usepackage[most]{tcolorbox}

\newtcbox{\mymath}[1][]{%
    nobeforeafter, math upper, tcbox raise base,
    enhanced, colframe=blue!60!black,
    colback=blue!20, boxrule=1pt,
    #1}


\usepackage{algorithm,algpseudocode}
\usepackage{algorithmicx}
\algrenewcommand\alglinenumber[1]{\footnotesize #1:} % Algorithm line number font size
\newcommand{\algFontSize}{\footnotesize}

\usepackage{tcolorbox}
\usepackage[english]{babel}
\usepackage{amsmath}
\usepackage{amsfonts}
\usepackage{fullpage}
\usepackage{graphicx}
\usepackage{setspace}
\usepackage{float}

\usepackage{tabu}% to increase table row separation

%\usepackage{showkeys}
\usepackage{multirow}

\usepackage{fancyvrb}% fancy verbatim

% \usepackage{mdframed}% for shaded framed boxes


% ************ ALLOW displayed equations to cross pages *********************
\allowdisplaybreaks

% ---------------------------- DATE HEADER ------------------------------------
% \usepackage{fancyhdr}
% \usepackage{time}
% take the % awaNOy on next line to produce the final camera-ready version
% Be sure to remove \thispagestyle{fancy} as well after the \make.
% \pagestyle{empty}
%
%\pagestyle{fancy}
% \newcommand\myTime{\now}
%\fancyhead{}
%\fancyhead[CO, CE]{\texttt{-Draft-}}
%\fancyhead[RO, RE]{\texttt{\today, \myTime}}
%\setlength{\headheight}{2\baselineskip}
%\renewcommand{\headrulewidth}{0pt}
% ---------------------------- DATE HEADER ------------------------------------



\usepackage{tikz}

\newtheorem{theorem}{Theorem}
\newtheorem{assumption}{Assumption}
\newtheorem{definition}{Definition}
\newenvironment{proof}[1][Proof]{\begin{trivlist}
\item[\hskip \labelsep {\bfseries #1.}]}{\end{trivlist}}

\newtheorem{thm}{Theorem}[section]
\newtheorem{prop}{Proposition}[section]
\newtheorem{lem}{Lemma}[section]
%\newtheorem{cor}{Corollary}[section]
\newtheorem{example}{Example}[section]
\newtheorem{conj}{Conjecture}[section]
\newtheorem{remark}{Remark}[section]
% some definitions of bold math italics to make typing easier.
% They are used in the corollary.

 
\input tex/defs.tex
\input tex/trimFig.tex
\usepackage{xargs}% for optional args to \newcommandx
\input tex/plotFigureMacros.tex

% \newcommand{\new}[1]{{\color{olive}#1}}
\newcommand{\new}[1]{{\color{red}#1}}
\newcommand{\wdh}[1]{{\color{DarkBlue}wdh: #1}}
\newcommand{\old}[1]{{\color{red}old: #1}}
\newcommand{\newer}[1]{{\color{orange} #1}}
\newcommand{\avk}[1]{{\color{orange}AVK: #1}}

\newcommand{\Ttilde}{\tilde{T}}


\newcommand{\pnl}{\p_{n_L}}
\newcommand{\pnr}{\p_{n_R}}
\newcommand{\mm}{m}% interface index character

\renewcommand\floatpagefraction{.99}
\renewcommand\topfraction{.99}
\renewcommand\bottomfraction{.99}
\renewcommand\textfraction{.01}   
\setcounter{totalnumber}{50}
\setcounter{topnumber}{50}
\setcounter{bottomnumber}{50}

% ============================================================================================
\begin{document}

\begin{frontmatter}
 \title{Notes on EVE: EigenVector Expansion Solver}


\author[rpi]{Jeffrey W.~Banks}
\ead{banksj3@rpi.edu}

\author[rpi]{William D.~Henshaw}
\ead{henshw@rpi.edu}

% \author[rpi]{Donald~W.~Schwendeman\fnref{NSFgrants}}
% \ead{schwed@rpi.edu}


\address[rpi]{Department of Mathematical Sciences, Rensselaer Polytechnic Institute, Troy, NY 12180, USA}

% \cortext[cor]{Corresponding author}

% \fntext[DARPA]{This work was partially funded by the DARPA Defense Sciences Office, Award HR00111720032.}
% \fntext[NSFgrants]{Research supported by the National Science Foundation under grants DMS-1519934 and DMS-1818926.}


% \fntext[RTG]{This work was partially funded by the NSF Research Training Group Grant DMS-1344962.}
% \fntext[DOE]{This work was partially performed under DOE contracts from the ASCR Applied Math Program.}
%\fntext[DOD]{This work was partially performed under DOD contract W911NF-14-C-0161.}

% \fntext[NSFgrantNew]{Research supported by the National Science Foundation under grant DMS-1519934.}

% \fntext[PECASEThanks]{Research supported by a U.S. Presidential Early Career Award for Scientists and Engineers.}

\begin{abstract}

We consider the numerical solution of initial-boundary-value problems IBVP's (or IVP's or Helmholtz problems )
using an eigenfunction expansion approach.
Discrete approximations to eigenfunctions for a given domain $\Omega$ are first computed. These eigefunctions are then used to
represent the general solution. The solution at any time can then be found.


\end{abstract}

\end{frontmatter}


% ------------- Table of contents
\tableofcontents

% -------------------------------------------------------------------------------------
\section{Introduction} 

We consider the numerical solution of initial-boundary-value problems IBVP's (or IVP's or Helmholtz problems )
using an eigenfunction expansion approach.
Discrete approximations to eigenfunctions for a given domain $\Omega$ are first computed. These eigefunctions are then used to
represent the general solution. The solution at any time can then be found.

% ----------------------------------------
\subsection{EVE for the Wave Equation}

Consider for example, the solution to the wave-like equation on a domain $\Omega$, with some initial
conditions and homogeneous boundary conditions,
\bse
\bat
  &  \p_t^2 u + L u =0 ,       \quad&& \xv\in\Omega, ~ t>0 \\
  &  u(\xv,0) = u_0(\xv),   \quad&& \xv\in\Omega,       \\ 
  &  \p_t u(\xv,0) = u_1(\xv),   \quad&& \xv\in\Omega,  \\
  &  \Bc u(\xv,t) = 0         \quad&& \xv\in\p\Omega,
\eat
\ese
where $L=-c^2\Delta$, for example. We want to initially consider problems where there is a basis
of ortho-normal eigenfunctions.
Suppose we know the eigenvalues and eigenfunctions 
\bse
\bat
  &   L \phi_j = \lambda_j \phi_j,        \quad&& \xv\in\Omega \\
  &  \phi_j(\xv) = 0                     \quad&& \xv\in\p\Omega,
\eat
\ese
The solution be represented as an eigenfunction expansion,
\ba
    u(\xv,t) = \sum_{j=0}^\infty q_j(t) \phi_j(\xv),
\ea
If the eigenfunctions are orthonormal, then
\ba
   q_j'' = - \lambda_j q_j
\ea
and
\ba
   \sum_{j=0}^\infty q_j(0) \phi_j(\xv)  = u_0(\xv), \\
   \sum_{j=0}^\infty q_j'(0) \phi_j(\xv)  = u_1(\xv)
\ea
If we know the eigenfunction expansions for the initial conditions,
\ba
   & u_0(\xv) = \sum_j \uHat_j \phi_j(\xv),  \\
   & u_1(\xv) = \sum_j \vHat_j \phi_j(\xv),  
\ea
then
\ba
   q_j(t) = \uHat_j \cos(\sqrt{\lambda_j} t) + \f{1}{\sqrt{\lambda_j}} \vHat_j \sin(\sqrt{\lambda_j} t)
\ea
Thus the solution is
\ba
  u(\xv,t) = \sum_j \Big( \uHat_j \cos(\sqrt{\lambda_j} t) + \f{1}{\sqrt{\lambda_j}} \vHat_j \sin(\sqrt{\lambda_j} t)\Big)
          \phi_j(\xv) \label{eq:eigenfunctionSolution}
\ea
Note that if the initial conditions $u_0(\xv)$ and $u_1(\xv)$ 
are smooth and satisfy the boundary conditions then we expect their
generalized Fourier series to converge rapidly (depending also on the smoothness of the domain).

% ------------------------------------------------------------
\input tex/forcedProblem



\input tex/boundaryConditions


% ********************************************
% \bogus
% {

% -------------------------------------------------------------------------------------
\input tex/parallelInTime 



% ==========================================================================================================
\input tex/numericalExamples

% } 



% \clearpage
% \input tex/oldStuff

% \bibliographystyle{elsart-num}
% \bibliography{journal-ISI,henshaw,henshawPapers}

\end{document}

% ***************************************************************************************************************
% ***************************************************************************************************************
% ***************************************************************************************************************
% ***************************************************************************************************************
% ***************************************************************************************************************

